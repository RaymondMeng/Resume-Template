\documentclass{muratcan_cv}
\usepackage {ctex}
\usepackage[utf8]{inputenc}
\setname{\textbf{\fontsize{22pt}{24pt}孟成}}{}
\setaddress{湖北省武汉市}
\setmobile{13696786574}
\setmail{raymondmeng@foxmail.com}
\setposition{本科三年级} %ignored for now
\setbloguarl{http://raymoncy.cn/} %you can play with color of the template (red is also nice..)
\setgithubaccount{https://github.com/RaymondMeng} %you can play with color of the template (red is also nice..)
\setthemecolor{red} %you can play with color of the template (red is also nice..)

\begin{document}
%Set variables
%You can add sections, texts, explanations just by copying the style below. Replace the dummy texts "\lipsum[1][x-x]\par" with actual texts.
%Create header
\headerview
\vspace{1ex}
%Sections
%
% Summary
\addblocktext{Summary}{%
 \fontsize{9pt}{24pt}对未知事物有着强烈的好奇心,有自我驱动的学习力和找到问题答案的能力。一直在奔跑,前往下一个未知点,不断探索!%replace this part with actual text
}
%
%Education
\section{Education} 
    \datedexperience{\fontsize{10pt}{24pt}武汉理工大学 Wuhan University Of Technology}{2019.9-至今} 
    \explanation{\fontsize{9pt}{24pt}通信工程,本科三年级在读} 
     \explanationdetail{\coloredbullet\ % 
    {\fontsize{9pt}{24pt}获院三好学生,担任校电子科技协会副会长,GPA:3.327,CET-4}
     }
     \explanationdetail{\coloredbullet\ % 
    {\fontsize{9pt}{24pt}\textbf{主攻方向:嵌入式应用、Cortex-M系列内核、FPGA逻辑设计、Hbirdv2-E203 RISC-V内核}} 
     }
     \explanationdetail{\coloredbullet\ % 
    {\fontsize{9pt}{24pt}\textbf{期望研究方向:数字IC/SOC设计、可重构计算、并行加速、超标量处理器设计}} 
     }
%
% Experience
\section{Experience}
    %
    \datedexperience{\fontsize{10pt}{24pt}基于机器视觉的管道焊缝检测系统}{2021} 
    \explanationdetail{\coloredbullet\ %
     \fontsize{9pt}{24pt}利用训练好的mobilenet-yolov3神经网络识别焊缝质量,并实现嵌入式装置在管道上自动运行并实现上位机反馈
     }
    %
    \datedexperience{\fontsize{10pt}{24pt}基于平头哥的智能无线头盔设计}{2021} 
    \explanationdetail{\coloredbullet\ %
     \fontsize{9pt}{24pt}利用平头哥WIFI RISC-V开发板与Digilent FPGA进行交互,完成环境数据的无线传输、无线图传等功能 
     }
    %
    \datedexperience{\fontsize{10pt}{24pt}用电器分析检测装置}{2021} 
    \explanationdetail{\coloredbullet\ %
     \fontsize{9pt}{24pt}利用电表芯片完成电网环境的检测分析,并通过数据处理分析用电器种类和数量,具有学习能力
     }
    %
    \datedexperience{\fontsize{10pt}{24pt}多模态智能警务机器人}{2021-2022} 
    \explanationdetail{\coloredbullet\ %
     \fontsize{9pt}{24pt}通过实时识别目标人物的情绪、面目表情,并且提取脸部特征等多模态数据,配合服务器进行大数据分析比对,甄辨嫌疑人 
     }
    %
    \datedexperience{\fontsize{10pt}{24pt}一款配备Cache、支持RV64IM的RISC-V流水线的处理器设计(IC全栈式设计)}{2022.2-至今} 
    \explanationdetail{\coloredbullet\ %
     \fontsize{9pt}{24pt}参加中科院组织的“一生一芯”项目计划并通过预学习答辩,成为项目正式成员,项目进度目前处在单周期处理器设计阶段 
     }
%
% Skills
\section{Skills}
    %
    \newcommand{\skillone}{\createskill{\fontsize{9pt}{24pt}编程语言}{\textbf{\emph{Experienced:}} \ \  C/C++ \cpshalf Verilog HDL \cpshalf Python \ \ \textbf{\emph{Familiar:}} \ \  Bash \cpshalf Chisel \cpshalf System Verilog}}
    %
    \newcommand{\skilltwo}{\createskill{\fontsize{9pt}{24pt}开发平台}{Windows/Linux \cpshalf STM32 \cpshalf ESP系列 \cpshalf MSP432 \cpshalf FPGA/ZYNQ \cpshalf Raspberry Pi/Lichee Pi \cpshalf K210 \cpshalf Jetson Nano \cpshalf TXW8301}}
    %
    \newcommand{\skillthree}{\createskill{\fontsize{9pt}{24pt}语言能力}{\fontsize{9pt}{24pt}英语具有良好的读写能力、通过CET-4}}
    %
    \createskills{\skillone, \skilltwo, \skillthree}
% Experience
\section{Achievements}
    \newcommand{\extraone}{%
    {\fontsize{9pt}{24pt}武汉理工大学校三等奖学金、院三好学生} \hfill \datetext{2020}%replace this part with actual text
    }
    %
    \newcommand{\extratwo}{%
    {\fontsize{9pt}{24pt}全国大学生工程训练与综合能力竞赛 \qquad\qquad\qquad\qquad\qquad \textbf{全国二等奖}} \hfill \datetext{2021}%replace this part with actual text
    }
    %
    \newcommand{\extrathree}{%
    {\fontsize{9pt}{24pt}"TI杯"全国大学生电子设计竞赛 \qquad\qquad\qquad\qquad\qquad\qquad\ \, \textbf{全国一等奖}} \hfill \datetext{2021}%replace this part with actual text
    }
    %
    \newcommand{\extrafour}{%
    {\fontsize{9pt}{24pt}第五届全国大学生集成电路创新创业大赛 \qquad\qquad\qquad\quad \ \ \textbf{全国三等奖}} \hfill \datetext{2021}%replace this part with actual text
    }
    %
    \newcommand{\extrafive}{%
    {\fontsize{9pt}{24pt}中国高校计算机大赛人工智能创意赛 \qquad\qquad\qquad\qquad\qquad \textbf{华中三等奖}} \hfill \datetext{2021}%replace this part with actual text
    }
    %
    \newcommand{\listofextras}{\extraone, \extratwo, \extrathree, \extrafour, \extrafive}
    %
    \createbullets{\listofextras}
%
%Footnote
\createfootnote
\end{document}
